% Options for packages loaded elsewhere
\PassOptionsToPackage{unicode}{hyperref}
\PassOptionsToPackage{hyphens}{url}
%
\documentclass[
]{article}
\title{Manuscript for Markup Language course - Using bayesian statistics
to prove a point}
\author{Inan Bostanci}
\date{1/15/2022}

\usepackage{amsmath,amssymb}
\usepackage{lmodern}
\usepackage{iftex}
\ifPDFTeX
  \usepackage[T1]{fontenc}
  \usepackage[utf8]{inputenc}
  \usepackage{textcomp} % provide euro and other symbols
\else % if luatex or xetex
  \usepackage{unicode-math}
  \defaultfontfeatures{Scale=MatchLowercase}
  \defaultfontfeatures[\rmfamily]{Ligatures=TeX,Scale=1}
\fi
% Use upquote if available, for straight quotes in verbatim environments
\IfFileExists{upquote.sty}{\usepackage{upquote}}{}
\IfFileExists{microtype.sty}{% use microtype if available
  \usepackage[]{microtype}
  \UseMicrotypeSet[protrusion]{basicmath} % disable protrusion for tt fonts
}{}
\makeatletter
\@ifundefined{KOMAClassName}{% if non-KOMA class
  \IfFileExists{parskip.sty}{%
    \usepackage{parskip}
  }{% else
    \setlength{\parindent}{0pt}
    \setlength{\parskip}{6pt plus 2pt minus 1pt}}
}{% if KOMA class
  \KOMAoptions{parskip=half}}
\makeatother
\usepackage{xcolor}
\IfFileExists{xurl.sty}{\usepackage{xurl}}{} % add URL line breaks if available
\IfFileExists{bookmark.sty}{\usepackage{bookmark}}{\usepackage{hyperref}}
\hypersetup{
  pdftitle={Manuscript for Markup Language course - Using bayesian statistics to prove a point},
  pdfauthor={Inan Bostanci},
  hidelinks,
  pdfcreator={LaTeX via pandoc}}
\urlstyle{same} % disable monospaced font for URLs
\usepackage[margin=1in]{geometry}
\usepackage{longtable,booktabs,array}
\usepackage{calc} % for calculating minipage widths
% Correct order of tables after \paragraph or \subparagraph
\usepackage{etoolbox}
\makeatletter
\patchcmd\longtable{\par}{\if@noskipsec\mbox{}\fi\par}{}{}
\makeatother
% Allow footnotes in longtable head/foot
\IfFileExists{footnotehyper.sty}{\usepackage{footnotehyper}}{\usepackage{footnote}}
\makesavenoteenv{longtable}
\usepackage{graphicx}
\makeatletter
\def\maxwidth{\ifdim\Gin@nat@width>\linewidth\linewidth\else\Gin@nat@width\fi}
\def\maxheight{\ifdim\Gin@nat@height>\textheight\textheight\else\Gin@nat@height\fi}
\makeatother
% Scale images if necessary, so that they will not overflow the page
% margins by default, and it is still possible to overwrite the defaults
% using explicit options in \includegraphics[width, height, ...]{}
\setkeys{Gin}{width=\maxwidth,height=\maxheight,keepaspectratio}
% Set default figure placement to htbp
\makeatletter
\def\fps@figure{htbp}
\makeatother
\setlength{\emergencystretch}{3em} % prevent overfull lines
\providecommand{\tightlist}{%
  \setlength{\itemsep}{0pt}\setlength{\parskip}{0pt}}
\setcounter{secnumdepth}{-\maxdimen} % remove section numbering
\ifLuaTeX
  \usepackage{selnolig}  % disable illegal ligatures
\fi

\begin{document}
\maketitle

\hypertarget{introduction}{%
\section{1 Introduction}\label{introduction}}

In this report, I want to inspect whether protein-consumption is a
better predictor of the average height of a country than the general
wealth of that country. A few weeks ago, I had a discussion with a
friend about determinants of height. I argued that the average height in
most western countries is larger than in other countries, mostly because
western countries consume the largest amount of protein, mostly in the
form of meat. My friend argued that western countries are wealthier and
therefore have better overall circumstances (i.e.~health, lifestyle)
which, together with genetics, are the biggest contributors and that
protein plays little to no role. I do believe that genetics and overall
circumstances have some effect, but I believe that protein is the more
important determinant. After all, animals kept in mass stocks, which
live under horrible circumstances but get fed tons of soy seem to grow
well and fast for profit-oriented industries. Because this assignment
came up soon after our discussion, I figured that it would be a good
idea to be the biggest know-it-all of all times by reminding my friend
about our discussion months later and showing him evidence for my
claim.\\
To answer the research question, I will compare several linear models by
means of the Deviance information criterion (DIC) and Bayes Factor.
Parameter estimates will be acquired in a Bayesian way, using a Marov
Chain Monte Carlo (MCMC) method, namely Gibbs sampling and
Metropolis-Hastings (MH) algorithms. The question will be answered by
both the parameter estimates and the model comparisons. This report is
outlined as follows: First, I will present the dataset and report
descriptive statistics. Next, I will elaborate on the method and explain
the algorithm. I will then assess the convergence of the model, test an
assumption of the model by means of a posterior predictive p-value (PPP)
and report the posterior means, standard deviations (SD) and the
95\%-central credible interval (CCI). After interpreting the estimates,
I will create two competing models and compare all models by means of
the DIC and Bayes Factor. Because of the less known process of Bayesian
statistics, I will also compare this method to a Frequentist approach.
Finally, I will make a conclusion on the research question.

\hypertarget{hypotheses}{%
\section{2 Hypotheses}\label{hypotheses}}

\(H_1\): Protein-consumption has a larger effect-size than wealth.\\
This will be tested by comparing the parameter estimates in a
standardized linear regression model containing both parameters.\\
\(H_2\): A model that contains only protein has a better model fit than
a model that contains only GDP.\\
It can be well-argued that protein-consumption and wealth themselves are
linearly related. Because protein is an expensive food, citizens in
wealthier countries are more likely to be able to afford protein.
However, if protein was just a noisy proxy for wealth, then a model with
wealth only should fit better than a model with protein only. If both
models fit equally well, it indicates that they are perfect substitutes.
I expect that a model with protein as the only predictor has a better
model fit than a model with wealth as the only predictor.

\hypertarget{data}{%
\section{3 Data}\label{data}}

The data was drawn from several datasets on the website Our World in
Data (Our World in Data 2021a, Our World in Data 2021b) and merged using
the country and year. The data contains information on each country for
GDP per capita (in dollars, adjusted for purchasing power), mean male
height (in cm) and the mean of consumed calories from animal protein.
Observations were made for each year from 1861 to 2019 although the
majority of years have missing values. Because the year 1996 contained
the lowest number of missing values, it was chosen for this study. This
dataset includes 171 countries. After removing rows with missing
observations, the final sample size is 158. Mean male height ranges from
159.9 to 182.5 with a mean of 171.3. GDP ranges from 262.5 to 101754.3
with a mean of 12676.6, median of 6806.4. The mean number of calories
from protein consumed per country ranges from 15.84 to 305.60 with a
mean of 127.37.

\hypertarget{method}{%
\section{4 Method}\label{method}}

\hypertarget{acquiring-the-model}{%
\subsection{4.1 Acquiring the model}\label{acquiring-the-model}}

The first hypothesis will be inspected by means of the linear regression
model
\[MeanHeight_i = \beta_0 + \beta_1*Protein_i + \beta_2 * GDP_i +  \varepsilon_i,\]
where \(\varepsilon \sim {\sf N}(0, \sigma^2)\).\\
Model estimates will be acquired using an MCMC-method that combines the
Gibbs sampling algorithm and the MH-algorithm. The Gibbs algorithm
iteratively samples a parameter from a conditional posterior
distribution of said parameter given the sampling distribution. The MH
algorithm does not sample from a conditional distribution. In short, it
iteratively samples from a proposal distribution that loosely
approximates the conditional posterior distribution and compares the
density of the conditional posterior distribution at the point of the
sampled value to the density of the conditional posterior distribution
at the point of the previously sampled value, while taking the
likelihood of drawing that sample given the proposal distribution into
account. The MH algorithm is applied for \(\beta_1\) (the coefficient
for protein), while the Gibbs sampling algorithm is applied each for
\(\beta_0\) (the coefficient of the intercept), \(\beta_2\) (the
coefficient for GDP) and the \(\sigma^2\) (the residual error
variance).\\
Because at the point of doing this study I only have an intuition of the
direction of the effects and their proportionality but no actual
historical data or grounded knowledge, the prior probabilities of the
coefficients will be uninformative. Therefore, they will each be
normally distributed with a mean of zero and a large variance of 1000:
\begin{align*}
b_0 \sim {\sf N}(0,1000) \\
b_j \sim {\sf N}(0,1000)
\end{align*} Since the outcome variable is also normally distributed,
the priors are conjugate, and the conditional posterior probabilities
are also normally distributed.\\
My prior the distribution of the residual error variance is proportional
to an inverse-gamma-distribution with shape and scale parameters each
approximating 0:
\[p(s^2) \propto \frac{1}{s^2} \approx IG(0.001, 0.001)\] To sample one
estimate from a conditional distribution, the parameters of that
conditional distribution have to be known. Therefore, these MCMC methods
require starting values. I will specify starting values for \(b_1\),
\(b_2\) and \(s^2\) and use these to sample the first value for b0,
which will then be used for the first sampled value of \(b_1\), etc. To
rule out that the starting value influences the estimated parameters,
the sampling procedure will be run twice with equal amounts of
iterations, each run being a separate chain. The starting values will be
considered as non-influential if the chains converge. Convergence will
be assessed by means of a visual inspection, a comparison of the mean
and standard deviation of each chain, autocorrelation and the Monte
Carlo error.

\hypertarget{regression-assumptions}{%
\subsection{4.3 Regression assumptions}\label{regression-assumptions}}

One assumption of a linear regression model is homoscedasticity,
i.e.~that the variance of the outcome variable is independent of the
predictors. A violation of this assumption can be pictured as a
heteroscedasticity in the residuals. I will test homoscedasticity by
comparing the correlation between the residuals and the predictor
variables with a PPP. First, I will compute the residuals of each set of
sampled parameters on the observed dataset. In an iterative procedure, I
will then simulate a dataset with each set of sampled parameters. Next,
I will correlate the fitted values with the residuals of each simulated
dataset and of the observed dataset, respectively. This correlation will
be the discrepancy measure of the test. If there is no clear trend about
the proportion of the correlation with the observed vs.~with the
simulated data, the residuals of the observed data seem to behave as
would be expected with the proposed model (represented by a PPP-value
close to 0.5). If the correlation appears to be larger in the observed
data than in the simulated data with most of the sets of parameters, the
variance of the outcome variable seems to be more dependent on the
predictors than would be expected with the proposed model (represented
by a PPP-value below 0.5).\\
I will also test the assumption of the absence of outliers with a PPP.
For this, I will use the same simulated datasets to compute the
difference between the largest and the smallest value as the
test-statistic. If the expected difference is smaller in the simulated
data, this indicates that the observed data is spread wider than
expected by the proposed model. If this is the case, I will perform two
additional tests using the difference between the mean and the largest
and lowest value, respectively, to inspect in which direction the
outliers are.

\hypertarget{testing-the-model-parameters}{%
\subsection{4.4 Testing the model
parameters}\label{testing-the-model-parameters}}

To test if the parameters are meaningful, the Bayes Factor (BF) will be
used for the hypothesis \(H_1\) of both coefficients being equal to zero
vs.~the complementary hypothesis \(H_u\) that the coefficients are not
equal to zero (\(H_1: \beta_1 = 0, \beta_2 = 0; H_u: \beta_1,\beta_2\)).
The BF can be interpreted in the following way: A BF of 0.5 indicates
that there is twice as much support for \(H_u\) than for \(H_1\), while
a BF of 2 indicates that there is twice as much support for \(H_1\) than
for \(H_u\).

\hypertarget{testing-the-hypotheses}{%
\subsection{4.5 Testing the hypotheses}\label{testing-the-hypotheses}}

Finally, I will test my claims about the effect of protein and GDP on
height in several ways. First, I will test the hypothesis that the
effect of protein is larger than the effect of GDP using the BF with
standardized coefficients (denoted by the subscript
\(stz; H_2: \beta_{1stz} > \beta_{2stz}; H_u: \beta_{1stz} ≤ \beta_{2stz}\)).
Next, I will use the MCMC-method to model the data with a
one-predictor-model using just the protein or GDP-variable. I will then
compare all three models by means of the DIC, which is a function of the
model's fit and complexity. I expect the DIC of the model containing
protein only to be lower than the DIC of the model containing GDP only.
Furthermore, I expect the model containing protein only to have about
the same DIC as the model containing both parameters. This is because I
expect the fit to be about as good, since I believe that protein is by
far the stronger predictor, and the complexity to be just minorly
different, since the full model only has one added parameter given a
large sample size.

\hypertarget{frequentist-methods-vs.-bayesian-methods}{%
\section{5 Frequentist methods vs.~Bayesian
methods}\label{frequentist-methods-vs.-bayesian-methods}}

This study does not use classical statistical methods to answer the
research question and instead draws upon Bayesian methods. This might
surprise my friend and I will therefore highlight the advantages for
this specific research question.\\
For once, it uses a PPP to test a model assumption. This procedure
allows for a much more interpretable inspection of how data should
behave if the proposed model is true, which is crucial to understand if
the proposed model is actually applicable to the scenario that was
observed.\\
It also uses the BF instead of a hypothesis significance test. The
largest advantage of this is that the BF allows to test more
differential hypotheses in a convenient way. In this instance, it allows
to test if one predictor has a larger effect than another, which would
not have been doable with the classical t-test that is being used for
the majority of linear regressions. Therefore, it can become much
clearer who of us is right about predictors for height, and I can be an
even bigger know-it-all than with simple significance tests.\\
A second advantage of the Bayesian approach lies in the sampling
procedure. When coefficients are very close to zero, researchers often
look at confidence intervals and inspect weather zero falls within the
interval. However, confidence intervals are not very meaningful in
single studies. The applied sampling procedure allows for the use of a
CCI and a distribution of the parameters, which does give an actual
indication of how likely the value zero is for a given coefficient.
Although I do not have a precise idea of how the parameters will look,
the range of human height is much smaller than the range of calories
consumed and of GDP, which might lead to such small coefficients close
to zero. Here, a look at the CCIs and histograms can be very
informative.

\hypertarget{results}{%
\section{6 Results}\label{results}}

Figure 1 depicts the trace plots of the parameters for each chain after
the burn-in period. Visually, both chains seem to have an equal width,
which indicates that the sampled values have the same range and mean,
and thus the algorithm seems to follow the same trend despite different
starting values. For the case of the first beta-coefficient
(\(\beta_1\)), which was sampled using the MH-algorithm, fluctuations
point at a large autocorrelation.\\
Figure 2 shows autocorrelation plots for each parameter. Except for the
case of the residual error variance, the sampling procedure seems to
suffer from a high autocorrelation and therefore requires a large number
of samples to cover the entire posterior distribution. This is also
reflected by the relatively low acceptance rate of 21.1\% for the
MH-step in the sampling algorithm. However, the 50.000 iterations that
were set for this study should be enough to cover the posterior
distribution.

\begin{figure}
\centering
\includegraphics{manuscript_inan_files/figure-latex/first sampler-1.pdf}
\caption{\label{fig:figs}Figure 1: Trace plots of both chains for each
parameter.}
\end{figure}

\begin{figure}
\centering
\includegraphics{manuscript_inan_files/figure-latex/autocorrelation plots-1.pdf}
\caption{\label{fig:figs}Figure 2: Autocorrelation plots for the pooled
sampled parameters.}
\end{figure}

Table 1 shows the sample statistics of the pooled chain for each
parameter, as well as the MC-error. The latter further does not speak
against convergence, because each MC-error is much smaller than its
corresponding standard deviation. Because no diagnostic criterion speaks
against convergence, I will assume that the sampler converged. The
posterior mean of the intercept-coefficient is at 164.66 (SD = 0.541,
95\%-CCI = {[}163.606; 165.721{]}), while the coefficient for protein
has a mean of 0.0596 (SD = 0.00478, 95\%-CCI = {[}0.05; 0.0691{]}) and
the coefficient for GDP has a mean of -0.000072 (SD = 0.000024, 95\%-CCI
= {[}-0.00012; -0.000025{]}). respectively. The mean of the residual
error variance is 12.533 (SD = 1.439, 95\%-CCI = {[}10.028; 15.65{]}).

\begin{longtable}[]{@{}lrrrrr@{}}
\caption{Table 1: Sample statistics for the posterior distributions of
the parameters. 2.5\% and 97.5\% represent quantiles to obtain the
95\%-CCI.}\tabularnewline
\toprule
Pooled chain & Mean & SD & 2.5\% & 97.5\% & MC Error \\
\midrule
\endfirsthead
\toprule
Pooled chain & Mean & SD & 2.5\% & 97.5\% & MC Error \\
\midrule
\endhead
b0 & 164.658876 & 0.541385 & 163.605810 & 165.720534 & 0.001712 \\
b1 & 0.059568 & 0.004779 & 0.050359 & 0.069109 & 0.000015 \\
b2 & -0.000072 & 0.000024 & -0.000120 & -0.000025 & 0.000000 \\
sigma2 & 12.533275 & 1.439044 & 10.027727 & 15.649456 & 0.004551 \\
\bottomrule
\end{longtable}

The PPP-value for the assumption of homoscedasticity is 0.448, which
means that the residuals are slightly less homoscedastic in the observed
data than in the simulated data. However, given that this value is very
close to 0.5, this difference seems to be very minor and the observed
data seems to behave almost as would be expected with the proposed model
in the majority of the instances.\\
The PPP-value for the assumption of the absence of outliers is 0.048.
This means that the proposed model expects a wider spread than present
in the data, and it also means that there are no outliers in the
observed data.

The bayes factor for \(H_1 (\beta_1 = 0; \beta_2 = 0)\) has a value of
0.00, meaning there is virtually infinitely more support for the
hypothesis that both coefficients are not 0, which indicates that the
model fits the data. The bayes factor for
\(H_2 (\beta_{1stz}> \beta_{2stz})\) with the standardized coefficients
has a value \textgreater{} 1,000,000, which indicates that there is an
unseizable amount of larger support for the hypothesis that the
standardized coefficient for protein is larger than the standardized
coefficient for GDP.

The DIC for the model containing both parameters is 2537. For the model
containing protein only, it is 2559 and for the model containing GDP
only, it is 2856.

\hypertarget{interpretation}{%
\section{7 Interpretation}\label{interpretation}}

The given data was successfully modelled with the MCMC-method, which was
indicated by the first BF. The posterior means of the parameters of
interest indicate that the calories coming from protein do have an
effect on the mean height of men in a given country. Curiously, the
posterior mean of the beta-coefficient for GDP is negative (although
very small). This might be because money is often distributed unequally
within countries and therefore, in some countries with a large GDP, the
wealth is not accessible to most citizens resulting in a shorter mean
height. Another possible reason is the collinearity between protein and
GDP, resulting in unreliable distributions. However, it might also be
that the distribution of GDP is so skewed that estimates become
unstable. This can be solved by log-transforming GDP, which will be
discussed at the end of this report.\\
The BF's point in favor of my expectations. There are effects of both
GDP and protein on the mean height of men within a country, but the
effect of protein is most definitely larger than the effect of GDP. The
incredibly large BF almost screams that I should not even have had this
discussion with my friend. The DICs for the one-parameter models, which
do not suffer from collinearity, further show that protein seems to
explain the mean height better than GDP does. Altogether, all the
evidence points in favor of my argument.

\hypertarget{discussion}{%
\section{8 Discussion}\label{discussion}}

In this report, I investigated whether protein-consumption is a better
explanation for the mean height of a population than overall wealth.
Using a MCMC-method, the DIC and BF, I showed that protein has a a
larger effect than GDP. However, the strong relationship between GDP and
protein might cause predictors to be unreliable. To circumvent this, I
inspected both predictors individually, showing that a model containing
protein only has a better fit than a model containing DIC only.\\
Besides the collinearity between protein and GDP, another problem is the
very skewed distribution of GDP. Typically, this is circumvented in
economic studies by log-transforming GDP. After writing the sampler, I
also ran it with the log-transformed GDP-variable. Figure 3 shows the
trace plots of both chains for that model after 10000 iterations on the
left and histograms for the posterior distributions of the parameters on
the right. It is very obvious that the chains did not converge yet. The
most striking observation in the histograms is the very wide
distribution of the coefficient for protein and that the first three
plots look ``less'' normal than they should. If I specify a shorter
range for the protein-coefficient, the plots actually start to resemble
uniform distributions or become skewed or have two modes. Using the
log(GDP)-variable in a one-predictor-model, this striking difference
does not appear, but the sampler needs quite a while to converge, too
(See Appendix).\\
I assume that the difference stems from the fact that the log of GDP is
even stronger correlated with protein and this collinearity results in
autocorrelation and weird distributions and in a wider distribution of
the protein-coefficient. Strong collinearity seems to cause slower
convergence. However, it is still puzzling why the sampler converges
more slowly when using the log(GDP)-variable individually.

\begin{figure}
\centering
\includegraphics{manuscript_inan_files/figure-latex/mh function log - traceplots-1.pdf}
\caption{\label{fig:figs}Figure 3: Trace plots and histograms for the
model using log(GDP).}
\end{figure}

\includegraphics{manuscript_inan_files/figure-latex/mh function log - posterior distributions-1.pdf}

Figure 3: Trace plots and histograms for the model using log(GDP).

\hypertarget{sources}{%
\section{Sources}\label{sources}}

Our World In Data 2021 a: Total calories from animal protein vs.~Mean
male height, 1996.
\url{https://ourworldindata.org/grapher/share-of-calories-from-animal-protein-vs-mean-male-height}.
(last accessed 2 June 2021).

Our World in Data 2021 b: Meat consumption vs.~GDP per capita, 2017.
\url{https://ourworldindata.org/grapher/meat-consumption-vs-gdp-per-capita}.
(last accessed 2 June 2021).

\hypertarget{appendix}{%
\section{Appendix}\label{appendix}}

\includegraphics{manuscript_inan_files/figure-latex/Appendix - mh function log(gdp) without protein, first coefficient empty - traceplots-1.pdf}

\includegraphics{manuscript_inan_files/figure-latex/Appendix - mh function log(gdp) without protein, first coefficient empty - autocorrelation plots-1.pdf}

Appendix I: Trace plots and histograms for the model using log(GDP) as
the only predictor, first coefficient empty (i.e.~\(x_1\) is a vector of
0's).

\includegraphics{manuscript_inan_files/figure-latex/Appendix - mh function log(gdp) without protein, second coefficient empty - traceplots-1.pdf}

\includegraphics{manuscript_inan_files/figure-latex/Appendix - mh function log(gdp) without protein, second coefficient empty - autocorrelation plots-1.pdf}

Appendix II: Trace plots and histograms for the model using log(GDP) as
the only predictor, second coefficient empty (i.e.~\(x_2\) is a vector
of 0's).

\end{document}
